\begin{abstract}
Most of the popular modern web-development frameworks, like Node.js, Flask or the standard library of Go, handle the creation and management of a running back-end service, in a mostly abstracted high-level way that does not allow a developer much freedom of modification of the inherent system architecture of the server. Such an inflexible and abstracted, often plug-and-play, server implementation helps to facilitate web development by concealing the system-level design choices from the end user. The problem of blindly relying in a web framework without understanding its internal architecture is that it might not be the most suitable choice for a particular web application, which then ends up having unnecessarily bloated and difficult to maintain dependency-prone services. The aim of this paper is to explore the design space of a dependency-free, cloud-deployable back-end service purely written in C, through a literature review and an actual software implementation. In order to weigh the advantages and challenges, in terms of both performance and software development ease, of a highly abstracted back-end framework and a self-made low-level service.
\end{abstract}
