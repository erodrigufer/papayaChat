\section{Conclusion}
Creating a back-end service from the ground up gives the developer the freedom of choosing between a thread-oriented, process-oriented, pre-emptively or non-pre-emptively scheduled architecture. The chat service developed in this work consists of a mainly IO-bound workload, so that it benefits from a pre-emptively scheduled architecture which changes processes handling clients as soon as they block \cite{Kennedy2018}. Nonetheless, it must be acknowledged that the development productivity, particularly because of the debugging of syscalls with strace, is not high compared to a high-level framework with mature networking libraries.

Furthermore, although the software was exclusively tested on Debian-based platforms, the dependency restriction of only using the C standard library and compiling with GCC did not guarantee an entirely bug-free portability between Unix platforms. A different default initialization of the parameters of some syscalls by the compiler generated difficult to find sources of faulty and unequal behaviour between platforms. Therefore, even when reducing dependencies to a bare minimum, it is an illusion to think that fully portable code can easily be generated, so that to some extent the appeal and reasoning behind OS-virtualization (container management systems) can be better grasped.

On the other hand, since the data model used in this implementation is immutable and uncomplicated, the implementation of an asynchronous data management system with transactional guarantees delivers benefits due to the reduction of system dependencies. A deployment in the cloud or in a local system is very expeditious and does not require the management and configuration of a DBMS.

The software created in this project, distributed through a public repository with a AGPL license (GNU Affero General Public License), delivers a functional command-line chat application that gives the user the possibility to self-host its chat service and regain full control over the management of its instant messaging data and metadata. Furthermore, it allows its users to avoid vendor lock-in effects and the single points of failure of a centralized architecture like Signal and WhatsApp, since its minimal amount of dependencies facilitate a prompt native deployment in any Unix system. The permissive open source license of the project allows further collaboration in order to improve the application in regards such as end to end encryption, UX and further tests in more platforms among others.