\section{Introduction}
When using any of the most popular modern web development frameworks the inherent system architecture supporting multiple concurrent client connections is often concealed from the developer. Moreover, the system-level design choices to handle concurrent connections tend to be immutable, so that, for instance, a framework based on a single threaded event loop architecture, like Node.js, cannot be modified or configured to work in a multi-threaded or multi-procedural way. 

Another aspect that characterizes some of these frameworks, is that the developer ends up having many different library or packet dependencies from a diverse range of sources, which are sometimes fundamental to enable basic functionality or enhance the capabilities of the framework. With an increasing number of dependencies numerous issues can arise, e.g. mutual incompatibilities between different package versions, management of vulnerabilities and difficulties recreating the same behaviour of an application between the development and the production environments. 

Describe that a chat app is being used as an example implementation. Argue why the architectural design of the chat app is more IO-bound that CPU-bound \cite{Kennedy2018} and therefore works best with a traditional pre-emptive scheduler.

"If you have a program that is focused on IO-Bound work, then context switches are going to be an advantage. Once a Thread moves into a Waiting state, another Thread in a Runnable state is there to take its place. This allows the core to always be doing work. This is one of the most important aspects of scheduling. Don’t allow a core to go idle if there is work (Threads in a Runnable state) to be done.

If your program is focused on CPU-Bound work, then context switches are going to be a performance nightmare. Since the Thead always has work to do, the context switch is stopping that work from progressing. This situation is in stark contrast with what happens with an IO-Bound workload"\cite{Kennedy2018} (correct this since it is actually from part 1)

