\section{State of the art}
As mentioned previously, choosing C as the sole development language is a deliberate design decision, since it provides all possible concurrency primitives natively in Unix systems (threads, message queues, processes, file locks, etc.), while only requiring the standard library. This fact provides as much freedom as possible to experiment and explore diverse concurrent back-end architectures, while also making the code easily portable to all Unix systems.

Arguably, Go is very well suited to be used as a state of the art comparison to this paper's proposed implementation in C. First of all, Go was created at Google in 2010 by some of the computer science pioneers that originally came up with Unix and C at Bell Labs, so it is no surprise that Go has been described as a "C-like language" or as "C for the 21st Century" \cite{GoPL2015}. Furthermore, it was created with "built-in concurrency" to tackle modern large distributed infrastructure problems and it is currently widely used at all network traffic levels as a server side service provider \cite{Pike2012}. Therefore, it is a great candidate as a point of reference of how modern server side network concurrency can be handled, from which a totally different architecture based on blocking processes can be developed. 

If the main goal of this paper is to open a developer's eyes to the many different concurrency paradigms that can be used for server side development, then the philosophy of Go (and for that matter, also of other popular frameworks like Node.js) is the antithesis of this work, because these frameworks provide an inflexible architecture that handles concurrency. In the case of Go, the syntax to handle the creation of concurrent workloads (so-called "\textit{goroutines}") and of communication channels between the goroutines is so simple that an unaware or beginner programmer might be completely oblivious of the scheduling work being performed under the hood by the Go runtime, or even of the fact that its code is running concurrently. 

\subsection{Goroutines}
The idiomatic way of dealing with client connections in Go, either in an HTTP server or through solely raw TCP communication, is by spawning a new goroutine that handles each client concurrently \cite{Morsing2013_2}\cite{GoNet}\cite{GoHTTP}. From a software engineering perspective this is a very practical approach, since it elevates the level of abstraction that the programmer has to deal with, so that it is unnecessary to directly intervene in memory synchronization and the management of a thread pool. This should have as a consequence gains in developer productivity, with the trade-off that there is less design freedom. The pledge of Go is that the runtime will single-handedly manage the scheduling of goroutines in the most effective way possible and that goroutines are so lightweight that the developer should not worry upfront about the amount of goroutines that would simultaneously be spawned \cite{Cox-Buday2017}.

Goroutines are very lightweight concurrent subroutines  supervised by the Go \textit{runtime} in userspace. Their memory footprint is very small, the assigned memory by default is only a few kilobytes at their creation \cite{Cox-Buday2017}. From the perspective of the kernel goroutines are non-preemptive, i.e. they are not interrupted by the OS scheduler to run other goroutines. They have defined \textit{points of entry} where they can be suspended or activated by the runtime scheduler, which is entirely running in userspace. Since a context-switch between goroutines happens in userspace and the runtime decides which data should be persistent between goroutines, it is orders of magnitude faster than context-switching between OS threads \cite{Cox-Buday2017} or between OS processes \cite{Kerrisk2010}. A context-switch between OS threads or processes is a costly operation in terms of both the kernel-side data structures needed to maintain all threads and processes, and the operations performed in kernel space to make the transition happen.

\subsection{Runtime scheduler}
Each Go executable is compiled with its own statically linked runtime environment in charge of scheduling the goroutines, garbage collection and other tasks. The system model that describes the runtime scheduler consists of three main elements: all statically and dynamically called goroutines, a context and the OS threads where the goroutines are run. Goroutines are placed by the runtime in either the local queue of a context or in the global queue pending to be run by the context in one of the OS threads. The contexts are in charge of managing the scheduling of the goroutine queues.

Parallelism in the system is achieved by having multiple contexts, each using a different core of the processor through different OS threads, in order to run the  goroutines waiting in their queues. The runtime manages a set of working threads coupled with contexts and another set of idle threads.  If a goroutine performs a syscall that would block, e.g. listens for clients on a TCP socket, the overlying OS thread in which the context is executing the goroutine would also have to block. In this scenario, the blocking thread is decoupled from the context, so that the context can grab one of the idle threads and keep working with other non-blocking goroutines.

As long as the goroutines running in the contexts do not call a blocking system call, the different goroutines in the queues can be freely interchanged at the given \textit{points of entry} by the scheduler within the same set of OS threads. This, as previously stated, avoids a costly context-switch in kernel space. 

Nonetheless, blocking syscalls for networking are handled in a special way by the runtime. As previously stated, Go idiomatically creates a new goroutine for each client connected to a server. If the server were to have thousands of simultaneously connected clients and most of the clients were to call blocking system calls at the same time, it would then have to create a unique blocked OS thread for each client. This state would be very costly because every blocked client goroutine translates to one blocked OS thread, consequently defeating the Go's goal keeping context-switches primarily in userspace.

Therefore, Go handles network connections in a way that avoids using to many system resources. First, when a new connection in accepted its file descriptor is set in non-blocking mode, which means that if I/O is not possible in the network socket, it would return an error instead of automatically blocking. So now, when a goroutine tries to perform I/O in a network socket and it returns an error, the goroutine notifies a special perpetually running thread called the "\textit{netpoller}", which polls the status of network sockets\cite{Morsing2013_2}. The goroutine which could not properly perform its network operation is placed back on a queue, until it is notified by the netpoller that it is possible to perform I/O in the file descriptor.

-----------------------------------

If, another main discussion in the methodology would be the async own database implementation, it would be interesting to talk about locks in any db system, and use the data book as a major reference.

Might as well read and describe how nginx works ?????
