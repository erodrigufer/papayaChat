\section{State of the art}
As mentioned previously, choosing C as the sole development language is a deliberate design decision, since it provides all possible concurrency primitives natively in Unix systems (threads, message queues, processes, file locks, etc.) while only requiring the standard library. This fact provides as much freedom as possible to experiment and explore diverse concurrent back-end architectures, while also making the code easily portable to all Unix systems.

Arguably, Go is very well suited to be used as a state of the art comparison to this paper's proposed implementation. First of all, Go was created at Google by some of the computer science pioneers that originally came up with Unix and C at Bell Labs, so it is no surprise that Go has been described as a "C-like language" or as "C for the 21st Century" \cite{GoPL2015}. Furthermore, among other reasons, it was created with "built-in concurrency" to tackle modern large distributed infrastructure problems and it is currently widely used at all network capacity levels as the server side service provider \cite{Pike2012}. Therefore, it is a great candidate as a point of reference of how modern server side network concurrency is handled, from which a totally different architecture based on blocking processes can be developed. 

If the main goal of this paper is to open a developer's eyes to the many different concurrency paradigms that can be used for server side development, then the philosophy of Go (and for that matter, also of other popular frameworks like Node.js) is the antithesis of this work.

-----------------------------------

Describe how the systems architecture of other popular web frameworks is, for instance Go and its goroutines.

Describe the concurrency model of Golang, and its goroutines. Explicitly mentioning that most developers have no knowledge or even control, over how this is done \cite{Cox-Buday2017}. Make clear that since we are developing the chat app in C, it makes sense to compare it with Go, since Go is C for the 21st Century/ it was developed by the Bell Labs architects of C (maybe find a good reference from the Go philosophy statement from its creators that talks about this). And Go also being a concurrency-first language with a very simple syntax for handling concurrency.

If, another main discussion in the methodology would be the async own database implementation, it would be interesting to talk about locks in any db system, and use the data book as a major reference.

Might as well read and describe how nginx works ?????
% Describe a normal RESTful API in Go