\section{State of the art}
C is a good choice of language in order to experiment with system-level concurrent programming primitives, since all of the available primitives in any Unix kernel can be used inside a program developed in C and its portability should be possible. Therefore, it makes sense to take a look at how a reference server implementation in Go handles concurrent connections, for many reasons: Go is a very popular modern language used widely to provide back-end services, it was designed to provide a simple to implement concurrency model, and it was developed by some of the original pioneers who created C at Bell Labs, it is no surprise that it is described as a "C-like language" or as "C for the 21st century" \cite{GoPL2015}.
-----------------------------------

Describe how the systems architecture of other popular web frameworks is, for instance Go and its goroutines.

Describe the concurrency model of Golang, and its goroutines. Explicitly mentioning that most developers have no knowledge or even control, over how this is done \cite{Cox-Buday2017}. Make clear that since we are developing the chat app in C, it makes sense to compare it with Go, since Go is C for the 21st Century/ it was developed by the Bell Labs architects of C (maybe find a good reference from the Go philosophy statement from its creators that talks about this). And Go also being a concurrency-first language with a very simple syntax for handling concurrency.

If, another main discussion in the methodology would be the async own database implementation, it would be interesting to talk about locks in any db system, and use the data book as a major reference.

Might as well read and describe how nginx works ?????
% Describe a normal RESTful API in Go