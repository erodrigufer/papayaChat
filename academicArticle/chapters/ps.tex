\section{Problem statement}
After having seen how modern web frameworks provide their users with an immutable low level architecture to handle concurrent client connections, exemplified by Go's standard library implementation of an HTTP(s) server, it will be discussed in this paper what other Unix system primitives could potentially be used to handle concurrency on a server.

The topic is first handled with a comparison of the advantages and disadvantages of different architectures. Afterwards, specific challenges regarding the implementation of the chat application will be covered, namely: how to guarantee a secure continuous operation of the server daemon, after opening a public port, what has to be considered when implementing a dependency-free database in a concurrent distributed system and is the application more easily portable when only having the standard C library as a dependency and compiling only with GCC?

How do two different concurrency approaches like processes and goroutines compare performance-wise (CPU utilization) with a same implementation?

Is there a difference in the coefficient of variation of both concurrency models? Does one of them has a more stable CPU utilization than the other one?

