\section{Problem statement}
As previously stated in the introduction, there is ample literature that affirms that kernel-space context-switching in multi-procedural programs tends to have a worse CPU utilization performance than user-space context-switching, but none of the bibliographic sources provide experimental data to corroborate this, show how big the difference is or perform a benchmark for network-related concurrency handling \cite{2003Events}\cite{2005Threads}\cite{2013ContextSwitching}\cite{Cox-Buday2017}\cite{Kerrisk2010}.

This work strives to first design the architecture of a multi-procedural concurrent IM application written in C and, then, create a test that makes a performance comparison between the preemptively scheduled C application and an equivalent non-preemptively scheduled IM application in Go possible, to address the following research questions (RQ):

\textbf{RQ 1: How does the mean CPU usage of a multi-procedural C application compare to the CPU usage of a similar non-preemptively scheduled program in Go while handling concurrent network connections?} RQ1 serves as an experimental comparison of the performance of context-switching in user-space and in kernel-space. 

\textbf{RQ 2: How stable over a period of time is the CPU usage of the two models being compared, when receiving a constant load?} This question is relevant to more accurately be able to provision servers for high load scenarios.

\textbf{RQ 3: How portable across Unix platforms is the IM application developed in C with minimal dependencies?} In other words can the program be built and executed without any changes across different Unix platforms?

% How do two different concurrency approaches like processes and goroutines compare performance-wise (CPU utilization) with a same implementation?



