\section{Discussion}
\subsection{CPU usage and load handling stability}
The results, as depicted in fig. \ref{fig_mean_values}, show an almost linear increase in the CPU usage mean value in the Go implementation, while the multi-process C implementation considerably decreases its CPU utilization after 16 concurrent connections. Overall, the C implementation handles the load more efficiently at all levels of concurrent connections. The results might seem counter-intuitive and baffling at first glance, since many literature sources warn against the performance penalties incurred by handling context-switches in kernel-space \cite{Cox-Buday2017}\cite{Kerrisk2010}. Nonetheless, a comparison based on more accurately simulating the behaviour of a real network application, takes into consideration 
\begin{figure}[h]
	\centering
	\includegraphics[width=2.5in]{img/cv.pdf}
	%where an .eps filename suffix will be assumed under latex, 
	%and a .pdf suffix will be assumed for pdflatex; or what has been declared
	%via \DeclareGraphicsExtensions.
	\caption{The figure compares the development of the coefficient of variation depending on the number of concurrent connections being handled by the two different concurrency platforms.}
	\label{fig_cv}
\end{figure}
\subsection{Portability issues}
Even though, only portable Unix syscalls are used and the number of dependencies is reduced to the utmost minimum of GCC and the C standard library, some portability issues arise when deploying in a multi-platform fashion.

The backend was compiled with GCC and tested in two different Debian-based distros: Ubuntu and Kali Linux, the latter was a 32-bit system, and in FreeBSD 13.0. A single compilation difference between the two Debian platforms rendered the backend service completely useless in one instance. The root cause of the faulty behaviour was then established using a syscall and signals monitoring tool like \textit{strace} in a very cumbersome process. The bug was caused by a difference in the default flags assigned by the compiler to the syscall assigning a signal disposition.

The same code that worked flawlessly in Ubuntu listened for clients in FreeBSD using solely IPv6, which changed the behaviour of the application massively and required code refactoring to enforce IPv4 in a cross-platform fashion.

Thus, it is illusory to think that restricting the dependencies to the bare minimum of the C standard library and GCC will make the code perfectly compatible across Unix systems. Debugging unexplained behaviour will still be arduous.